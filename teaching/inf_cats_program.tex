\documentclass{amsart}
\usepackage{amscd}
\usepackage{color}
\usepackage{fontspec}
\usepackage{amsmath}
\usepackage{amsfonts}
\usepackage{amsthm}
\usepackage{amssymb}
\usepackage{bbm}
\usepackage{graphicx}
\usepackage{epstopdf}
\newcommand\hmmax{0}
\newcommand\bmmax{0}
\usepackage{bm}
\usepackage[all,cmtip]{xy}
\usepackage{csquotes}
\usepackage{hyperref}
\usepackage{enumerate}
\usepackage{enumitem}
\usepackage{mathrsfs}
\usepackage{vmargin}
\usepackage{verbatim}
\usepackage{cleveref}
\usepackage{stackrel}
\usepackage{epigraph}
\usepackage{chngcntr}
\usepackage{etoolbox}
\usepackage[backend=biber, doi=false,isbn=false, url=false]{biblatex}
\newtheorem{theo}{Theorem}[section]
\newtheorem{cor}[theo]{Corollary}
\newtheorem{prop}[theo]{Proposition}
\newtheorem{lemma}[theo]{Lemma}
\newtheorem{claim}[theo]{Claim}
\newtheorem{conj}[theo]{Conjecture}
\newtheorem{question}[theo]{Question}
\theoremstyle{definition}
\newtheorem{defi}[theo]{Definition}
\theoremstyle{remark}
\newtheorem{remark}[theo]{Remark}
\newtheorem{notation}[theo]{Notation}
\newcommand{\BA}{{\mathbb{A}}}
\newcommand{\BB}{{\mathbb{B}}}
\newcommand{\BC}{{\mathbb{C}}}
\newcommand{\BD}{{\mathbb{D}}}
\newcommand{\BE}{{\mathbb{E}}}
\newcommand{\BF}{{\mathbb{F}}}
\newcommand{\BG}{{\mathbb{G}}}
\newcommand{\BH}{{\mathbb{H}}}
\newcommand{\BI}{{\mathbb{I}}}
\newcommand{\BJ}{{\mathbb{J}}}
\newcommand{\BK}{{\mathbb{K}}}
\newcommand{\BL}{{\mathbb{L}}}
\newcommand{\BM}{{\mathbb{M}}}
\newcommand{\BN}{{\mathbb{N}}}
\newcommand{\BO}{{\mathbb{O}}}
\newcommand{\BP}{{\mathbb{P}}}
\newcommand{\BQ}{{\mathbb{Q}}}
\newcommand{\BR}{{\mathbb{R}}}
\newcommand{\BS}{{\mathbb{S}}}
\newcommand{\BT}{{\mathbb{T}}}
\newcommand{\BU}{{\mathbb{U}}}
\newcommand{\BV}{{\mathbb{V}}}
\newcommand{\BW}{{\mathbb{W}}}
\newcommand{\BX}{{\mathbb{X}}}
\newcommand{\BY}{{\mathbb{Y}}}
\newcommand{\BZ}{{\mathbb{Z}}}
\newcommand{\Fa}{{\mathfrak{a}}}
\newcommand{\Fb}{{\mathfrak{b}}}
\newcommand{\Fc}{{\mathfrak{c}}}
\newcommand{\Fd}{{\mathfrak{d}}}
\newcommand{\Fe}{{\mathfrak{e}}}
\newcommand{\Ff}{{\mathfrak{f}}}
\newcommand{\Fg}{{\mathfrak{g}}}
\newcommand{\Fh}{{\mathfrak{h}}}
\newcommand{\Fi}{{\mathfrak{i}}}
\newcommand{\Fj}{{\mathfrak{j}}}
\newcommand{\Fk}{{\mathfrak{k}}}
\newcommand{\Fl}{{\mathfrak{l}}}
\newcommand{\Fm}{{\mathfrak{m}}}
\newcommand{\Fn}{{\mathfrak{n}}}
\newcommand{\Fo}{{\mathfrak{o}}}
\newcommand{\Fp}{{\mathfrak{p}}}
\newcommand{\Fq}{{\mathfrak{q}}}
\newcommand{\Fr}{{\mathfrak{r}}}
\newcommand{\Fs}{{\mathfrak{s}}}
\newcommand{\Ft}{{\mathfrak{t}}}
\newcommand{\Fu}{{\mathfrak{u}}}
\newcommand{\Fv}{{\mathfrak{v}}}
\newcommand{\Fw}{{\mathfrak{w}}}
\newcommand{\Fx}{{\mathfrak{x}}}
\newcommand{\Fy}{{\mathfrak{y}}}
\newcommand{\Fz}{{\mathfrak{z}}}
\newcommand{\FA}{{\mathfrak{A}}}
\newcommand{\FB}{{\mathfrak{B}}}
\newcommand{\FC}{{\mathfrak{C}}}
\newcommand{\FD}{{\mathfrak{D}}}
\newcommand{\FE}{{\mathfrak{E}}}
\newcommand{\FF}{{\mathfrak{F}}}
\newcommand{\FG}{{\mathfrak{G}}}
\newcommand{\FH}{{\mathfrak{H}}}
\newcommand{\FI}{{\mathfrak{I}}}
\newcommand{\FJ}{{\mathfrak{J}}}
\newcommand{\FK}{{\mathfrak{K}}}
\newcommand{\FL}{{\mathfrak{L}}}
\newcommand{\FM}{{\mathfrak{M}}}
\newcommand{\FN}{{\mathfrak{N}}}
\newcommand{\FO}{{\mathfrak{O}}}
\newcommand{\FP}{{\mathfrak{P}}}
\newcommand{\FQ}{{\mathfrak{Q}}}
\newcommand{\FR}{{\mathfrak{R}}}
\newcommand{\FS}{{\mathfrak{S}}}
\newcommand{\FT}{{\mathfrak{T}}}
\newcommand{\FU}{{\mathfrak{U}}}
\newcommand{\FV}{{\mathfrak{V}}}
\newcommand{\FW}{{\mathfrak{W}}}
\newcommand{\FX}{{\mathfrak{X}}}
\newcommand{\FY}{{\mathfrak{Y}}}
\newcommand{\FZ}{{\mathfrak{Z}}}
\newcommand{\Ca}{{\mathcal{a}}}
\newcommand{\Cb}{{\mathcal{b}}}
\newcommand{\Cc}{{\mathcal{c}}}
\newcommand{\Cd}{{\mathcal{d}}}
\newcommand{\Ce}{{\mathcal{e}}}
\newcommand{\Cf}{{\mathcal{f}}}
\newcommand{\Cg}{{\mathcal{g}}}
\newcommand{\Ch}{{\mathcal{h}}}
\newcommand{\Ci}{{\mathcal{i}}}
\newcommand{\Cj}{{\mathcal{j}}}
\newcommand{\Ck}{{\mathcal{k}}}
\newcommand{\Cl}{{\mathcal{l}}}
\newcommand{\Cm}{{\mathcal{m}}}
\newcommand{\Cn}{{\mathcal{n}}}
\newcommand{\Co}{{\mathcal{o}}}
\newcommand{\Cp}{{\mathcal{p}}}
\newcommand{\Cq}{{\mathcal{q}}}
\newcommand{\Cr}{{\mathcal{r}}}
\newcommand{\Cs}{{\mathcal{s}}}
\newcommand{\Ct}{{\mathcal{t}}}
\newcommand{\Cu}{{\mathcal{u}}}
\newcommand{\Cv}{{\mathcal{v}}}
\newcommand{\Cw}{{\mathcal{w}}}
\newcommand{\Cx}{{\mathcal{x}}}
\newcommand{\Cy}{{\mathcal{y}}}
\newcommand{\Cz}{{\mathcal{z}}}
\newcommand{\CA}{{\mathcal{A}}}
\newcommand{\CB}{{\mathcal{B}}}
\newcommand{\CC}{{\mathcal{C}}}
\renewcommand{\CD}{{\mathcal{D}}}
\newcommand{\CE}{{\mathcal{E}}}
\newcommand{\CF}{{\mathcal{F}}}
\newcommand{\CG}{{\mathcal{G}}}
\newcommand{\CH}{{\mathcal{H}}}
\newcommand{\CI}{{\mathcal{I}}}
\newcommand{\CJ}{{\mathcal{J}}}
\newcommand{\CK}{{\mathcal{K}}}
\newcommand{\CL}{{\mathcal{L}}}
\newcommand{\CM}{{\mathcal{M}}}
\newcommand{\CN}{{\mathcal{N}}}
\newcommand{\CO}{{\mathcal{O}}}
\newcommand{\CP}{{\mathcal{P}}}
\newcommand{\CQ}{{\mathcal{Q}}}
\newcommand{\CR}{{\mathcal{R}}}
\newcommand{\CS}{{\mathcal{S}}}
\newcommand{\CT}{{\mathcal{T}}}
\newcommand{\CU}{{\mathcal{U}}}
\newcommand{\CV}{{\mathcal{V}}}
\newcommand{\CW}{{\mathcal{W}}}
\newcommand{\CX}{{\mathcal{X}}}
\newcommand{\CY}{{\mathcal{Y}}}
\newcommand{\CZ}{{\mathcal{Z}}}
\newcommand{\Hom}{\mathop{\rm Hom}\nolimits}
\newcommand{\Homint}{\underline{\mathsf{Hom}}}
\newcommand{\Homsh}{\mathcal{H}om}
\newcommand{\RHom}{\mathop{\rm RHom}\nolimits}
\newcommand{\Ext}{\mathop{\rm Ext}\nolimits}
\newcommand{\Extsh}{\mathcal{E}xt}
\newcommand{\YExt}{\mathop{\rm YExt}\nolimits}
\newcommand{\Lim}{{\rm Lim}}
\newcommand{\Colim}{{\rm Colim}}
\DeclareMathOperator*{\Holim}{{\rm Holim}}
\DeclareMathOperator*{\Hocolim}{{\rm Hocolim}}
\newcommand{\ra}{\rightarrow}
\newcommand{\lra}{\longrightarrow}
\newcommand{\rap}{\stackrel{+}{\rightarrow}}
\newcommand{\Spec}{\mathop{{\bf Spec}}\nolimits}
\newcommand{\Spm}{\mathop{{\bf Spm}}\nolimits}
\newcommand{\Spf}{\mathop{{\bf Spf}}\nolimits}
\newcommand{\Proj}{\mathop{{\bf Proj}}\nolimits}
\newcommand{\Th}{\mathop{{\bf Th}}\nolimits}
\newcommand{\Sch}{\mathsf{Sch}}
\newcommand{\Sm}{\mathsf{Sm}}
\newcommand{\AnSm}{\mathsf{AnSm}}
\newcommand{\Ouv}{\mathsf{Ouv}}
\newcommand{\SmProj}{\mathsf{SmProj}}
\newcommand{\un}{\mathds{1}}
\newcommand{\Be}{{\scriptsize \mbox{\foreignlanguage{russian}{B}}}}
\newcommand{\Ob}{\mathsf{Ob}}
\newcommand{\Gm}{\mathbb{G}_\mathrm{m}}
\newcommand{\Ga}{\mathbb{G}_\mathrm{a}}
\newcommand{\pointille}{{_.}^.}
\newcommand{\card}{\mathop{\rm card}\nolimits}
\newcommand{\trace}{\mathop{\rm Tr}\nolimits}
\newcommand{\ord}{\mathop{\rm ord}\nolimits}
\newcommand{\charact}{\mathop{\rm char}\nolimits}
\newcommand{\rank}{\mathop{{\rm rank}}\nolimits}
\newcommand{\Gal}{\mathop{\rm Gal}\nolimits}
\newcommand{\SH}{\mathop{\mathbf{SH}}\nolimits}
\newcommand{\DM}{\mathop{\mathbf{DM}}\nolimits}
\newcommand{\DA}{\mathop{\mathbf{DA}}\nolimits}
\newcommand{\AnDA}{\mathop{\mathbf{AnDA}}\nolimits}
\newcommand{\Chow}{\mathop{\mathbf{Chow}}\nolimits}
\newcommand{\HI}{\mathop{\mathbf{HI}}\nolimits}
\newcommand{\MM}{\mathop{\mathbf{MM}}\nolimits}
\newcommand{\Cpl}{\mathop{\mathbf{Cpl}}\nolimits}
\newcommand{\Spt}{\mathop{\mathbf{Spt}}\nolimits}
\newcommand{\PSh}{\mathop{\mathbf{PSh}}\nolimits}
\newcommand{\Sh}{\mathop{\mathbf{Sh}}\nolimits}
\newcommand{\Mod}{\rm Mod}
\newcommand{\Mon}{\rm Mon}
\newcommand{\CMon}{\rm CMon}
\newcommand{\Cat}{\rm Cat}
\newcommand{\id}{{\rm id}}
\newcommand{\Ho}{\mathbf{Ho}}
\newcommand{\PreShv}{\mathbf{PSh}}
\newcommand{\Shv}{\mathbf{Sh}}
\newcommand{\D}{\mathsf{D}}
\newcommand{\Sym}{\mathsf{Sym}}
\newcommand{\Coh}{\mathsf{Coh}}
\newcommand{\Alt}{\mathsf{Alt}}
\newcommand{\SmCor}{\mathsf{SmCor}}
\newcommand{\Var}{\mathsf{Var}}
\newcommand{\An}{{\rm An}}
\newcommand{\Pic}{{\rm Pic}}
\newcommand{\sPic}{\mathcal{P}ic}
\newcommand{\NS}{{\rm NS}}
\newcommand{\Alb}{{\rm Alb}}
\newcommand{\Div}{{\rm Div}}
\newcommand{\Aut}{{\rm Aut}}
\newcommand{\Nis}{{\rm Nis}}
\newcommand{\Et}{{\rm Et}}
\newcommand{\Zar}{{\rm Zar}}
\newcommand{\Bor}{Bor}
\newcommand{\GL}{{\rm GL}}
\newcommand{\SL}{{\rm SL}}
\newcommand{\PGL}{{\rm PGL}}
\newcommand{\Gr}{{\rm Gr}}
\newcommand{\image}{\mathop{{\rm Im}}\nolimits}
\newcommand{\imm}{\mathop{{\rm im}}\nolimits}
\newcommand{\coimage}{\mathop{{\rm coim}}\nolimits}
\newcommand{\Lie}{\mathop{\rm Lie}\nolimits}
\newcommand{\End}{\mathop{\rm End}\nolimits}
\newcommand{\Isom}{\mathop{\rm Isom}\nolimits}
\newcommand{\Mor}{\mathop{\rm Mor}\nolimits}
\newcommand{\tildeExt}{\widetilde{\rm Ext}\mathstrut}
\newcommand{\UHom}{\mathop{\underline{\rm Hom}}\nolimits}
\newcommand{\UAut}{\mathop{\underline{\rm Aut}}\nolimits}
\newcommand{\Cent}{\mathop{\rm Cent}\nolimits}
\newcommand{\Norm}{\mathop{\rm Norm}\nolimits}
\newcommand{\Stab}{\mathop{\rm Stab}\nolimits}
\newcommand{\Quot}{\mathop{\rm Quot}\nolimits}
\newcommand{\Res}{\mathop{\rm Res}\nolimits}
\newcommand{\Ind}{\mathop{\rm Ind}\nolimits}
\newcommand{\Frac}{\mathop{\rm Frac}\nolimits}
\newcommand{\Id}{\mathop{\rm Id}\nolimits}
\newcommand{\CoInd}{\mathop{\rm CoInd}\nolimits}
\newcommand{\Tot}{\mathop{\rm Tot}\nolimits}
\newcommand{\DTot}{\mathop{\rm DTot}\nolimits}
\newcommand{\Pro}{\mathop{\rm Pro}\nolimits}
\newcommand{\Sus}{\mathop{\rm Sus}\nolimits}
\newcommand{\LSus}{\mathop{\rm LSus}\nolimits}
\newcommand{\Ev}{\mathop{\rm Ev}\nolimits}
\newcommand{\REv}{\mathop{\rm REv}\nolimits}
\newcommand{\Frob}{\mathop{{\rm Frob}}\nolimits}
\newcommand{\Tor}{\mathop{\rm Tor}\nolimits}
\newcommand{\Coker}{\mathop{\rm Coker}\nolimits}
\newcommand{\Ker}{\mathop{\rm Ker}\nolimits}
\newcommand{\supp}{\mathop{\rm Supp}\nolimits}
\newcommand{\Jac}{\mathop{\rm Jac}\nolimits}
\newcommand{\df}{\mathrm{df}}
\newcommand{\JG}{\mathcal{JG}}
\newcommand{\DG}{\mathcal{DG}}
\newcommand{\CCG}{\mathcal{CG}}
\newcommand{\PG}{\mathcal{PG}}
\newcommand{\sNS}{\mathcal{NS}}
\newcommand{\NSL}{\mathcal{NSL}}
\newcommand{\Picsm}{\mathcal{P}ic^\sm}
\newcommand{\Picsmc}{\mathcal{P}ic^\sm_*}
\newcommand{\adj}{\mathop{\rm adj}\nolimits}
\newcommand{\basechange}{\rm base\ change}
\newcommand{\Lotimes}{\mathbin{\stackrel{L}{\otimes}}}
\newcommand{\loccit}{[loc.$\;$cit.]}
\newcommand{\OFU}{\overline{\FU}}
\newcommand{\Ug}{\underline{g}}
\newcommand{\Un}{\underline{n}}
\newcommand{\Ur}{\underline{r}}
\newcommand{\Ux}{\underline{x}}
\newcommand{\diag}{\mathop{\rm diag}\nolimits}
\newcommand{\pr}{\mathop{\rm pr}\nolimits}
\newcommand{\Cone}{{\rm Cone}}
\newcommand{\CHo}{\mathop{\rm CH}\nolimits}
\newcommand{\LAlb}{\mathop{\rm LAlb}\nolimits}
\newcommand{\RPic}{\mathop{\rm RPic}\nolimits}
\newcommand{\Ab}{\mathop{\rm Ab}\nolimits}
\newcommand{\For}{\mathop{\rm For}\nolimits}
\newcommand{\Set}{\mathop{\rm Set}\nolimits}
\newcommand{\corexp}{\mathrm{cor}}
\newcommand{\et}{\mathrm{\'et}}
\newcommand{\eff}{\mathrm{eff}}
\newcommand{\qfh}{\mathrm{qfh}}
\newcommand{\gm}{\mathrm{gm}}
\newcommand{\op}{\mathrm{op}}
\newcommand{\aug}{\mathrm{aug}}
\newcommand{\coh}{\mathrm{coh}}
\newcommand{\homo}{\mathrm{hom}}
\newcommand{\dcoh}{\mathrm{dcoh}}
\newcommand{\red}{\mathrm{red}}
\newcommand{\sm}{\mathrm{sm}}
\newcommand{\ssm}{\mathrm{ssm}}
\newcommand{\gsm}{\mathrm{gsm}}
\newcommand{\ins}{\mathrm{ins}}
\newcommand{\ind}{\mathrm{ins}}
\newcommand{\nc}{\mathrm{nc}}
\newcommand{\mode}{\mathrm{mod}}
\newcommand{\sep}{\mathrm{sep}}
\newcommand{\s}{\mathrm{s}}
\newcommand{\nr}{\mathrm{nr}}
\newcommand{\tor}{{\rm tor}}
\newcommand{\opp}{{\rm opp}}
\newcommand{\steff}{{\rm st-eff}}
\newcommand{\Ex}{{\rm Ex}}
\newcommand{\tr}{{\rm tr}}
\newcommand{\perf}{{\rm perf}}
\newcommand{\fr}{{\rm fr}}
\newcommand{\gr}{{\rm gr}}
\newcommand{\str}{{\rm str}}
\newcommand{\ab}{{\rm ab}}
\newcommand{\num}{{\rm num}}
\newcommand{\pure}{{\rm pure}}
\newcommand{\an}{{\rm an}}
\newcommand{\psh}{{\rm psh}}
\newcommand{\adjo}{{\rm adj}}
\newcommand{\TODO}{{\color{red} TODO }}
\newcommand{\REF}{{\color{green} REF }}
		                        \usepackage{stmaryrd}


\newcommand\cosimparrowone{%
        \mathrel{\vcenter{\mathsurround0pt
                \ialign{##\crcr
                      \noalign{\nointerlineskip}$\rightarrow$\crcr
                      \noalign{\nointerlineskip}$\leftarrow$\crcr
                      \noalign{\nointerlineskip}$\rightarrow$\crcr
                }%
        }}%
    }
\newcommand\cosimparrowtwo{%
        \mathrel{\vcenter{\mathsurround0pt
                \ialign{##\crcr
                        \noalign{\nointerlineskip}$\rightarrow$\crcr
                        \noalign{\nointerlineskip}$\leftarrow$\crcr
                        \noalign{\nointerlineskip}$\rightarrow$\crcr
                        \noalign{\nointerlineskip}$\leftarrow$\crcr
                        \noalign{\nointerlineskip}$\rightarrow$\crcr

                }%
        }}%
}
    
\addbibresource{DGT.bib}
\date{\today}
\title{Categories and infinity-categories}
\hypersetup{
 pdfauthor={},
 pdftitle={},
 pdfkeywords={},
 pdfsubject={},
 pdfcreator={Emacs 25.3.50.2 (Org mode 9.1.2)}, 
 pdflang={English}}
\begin{document}

\maketitle

\section*{Introduction}

Infinity category theory lies in the intersection of two major developments of 20th century mathematics: topology and category theory. Category theory is a very powerful framework to organize and unify mathematical theories. Infinity category theory extends this framework to settings where the morphisms between two objects form not a set but a topological space (or a related object like a chain complex). This situation arises naturally in homological algebra, algebraic topology and sheaf theory. 

This reading seminar will recall the foundational ideas of usual category theory and then make the transition to homotopical algebra and infinity categories. By the end of the seminar, the student will be familiar enough with infinity categories that they can navigate texts written in this new language.

\section*{Guidelines for the talks}

\begin{itemize}
\item The talks are given in English, should be given on the blackboard, and should last approximately 80 minutes to
allow for 10 minutes of questions.
\item Participants are expected to discuss their talk with me the week before they are scheduled to
speak (and bring with them a draft of their talk notes). The default appointment for this
discussion is Wednesday morning at 10am (the week before the talk) in room 108, Arnimallee 3 (if the participant
is not available at this time, they should email before this date to arrange a different time).
\item All required definitions and mathematical claims should be clearly stated; in particular, the
  definitions of all terms in italics in the descriptions below should be given.
\item The speaker should make sure that the assumptions and the claim are clear to the audience, in order for the
other participants to be able to follow proofs and explanations.
\end{itemize}

\section*{Program}

\subsection{October 18th: Introduction (Simon Pepin Lehalleur)}

Given by the lecturer.

\subsection{October 25th: Categories, Functors and natural transformations}

Following \cite[Chapter 1]{Riehl_context}, introduce the basic definitions of category theory.

\begin{itemize}
\item Skip the historical introduction before 1.1.
\item Present \S 1.1, with emphasis on some of the concrete examples of 1.1.3-4: Set, Top, Group, Ring, Mod$_R$, Man, Poset, Ch$_R$.
\item Present the three exercises of \S 1.1.
\item Explain categorical duality following \S 1.2.
\item Do some of the exercises of \S 1.2, according to your personal preference.
\item Explain the definition of a functor 1.3.1 and present some of the examples in 1.3.2: (i), (ii), (iii), (vi), (viii), (ix).
\item Present the rest of \S 1.3,skipping 1.3.3-4 and 1.3.15.
\item Explain the issue with duality of vector spaces at the beginning of \S 1.4.
\item Explain the definition of a natural transformation between functors 1.4.1; the rest of the section consists of examples, explain some in 1.4.3.
\item Give definitions 1.5.4, 1.5.7 and Theorem 1.5.9. Explain corollary 1.5.11.
\end{itemize}

\subsection{November 1st: Universal Properties, Representable Functors and the Yoneda Lemma}

Following \cite[Chapter 2]{Riehl_context}, discuss how category theory gives a powerful formalisation of the idea of ``universal properties''.

\begin{itemize}
\item Define representable functors and explain many examples as in \S 2.1. State the questions at the end of \S 2.1 as motivation for the Yoneda lemma.
\item State and prove the Yoneda lemma 2.2.4.
\item Explain corollary 2.2.8.
\item Present \S 2.3. The example 2.3.7 of tensor products, seen through the perspective of representable functors, is a very good one.
\item Present the definition 2.4.1-2 and explain their relation with slice categories as in 2.4.6.
\end{itemize}

\subsection{November 11th: Limits and Colimits}

Following \cite[Chapter 3]{Riehl_context}, discuss the fundamental notion of limits and colimits of functors.

\begin{itemize}
\item Explain all of \S 3.1, perhaps skipping a few examples, but essentially everything here is important!
\item Discuss how to construct all limits in the category of sets explicitely: 3.2.6 and 3.2.13.
\item Explain what it means for a functor to preserve (co)limits 3.3.1, skip the rest of \S 3.3.
\item Explain 3.4.2. State 3.4.6 and its dual 3.4.11 (without the reflection part). State Theorem 3.4.12.
\item Give the definition of complete and cocomplete categories, and some examples and counter-examples as in the beginning of \S 3.5.
\end{itemize}

\subsection{November 15th: Adjoint Functors}

Following \cite[Chapter 4]{Riehl_context}, discuss the adjoint functors and its relationship to all the category theory we know so far.

\begin{itemize}
\item Present the definition of adjoint functors and some of the examples of \S 4.1.
\item Explain units, counits and the alternative formulation of adjunction in \S 4.2.
\item Skip \S 4.3.
\item Present all the results in \S 4.4 and sketch at least one proof.
\item Explain the fundamental results 4.5.1-3 and its applications 4.5.7-11. Explain Lemma 4.5.13.
\end{itemize}




\subsection{November 22th: Simplicial sets}

A \emph{simplicial set} is an abstract combinatorial object, which allows to model topological space up to homotopy. We have a set $K_n$ for every $0 \leq n < \infty$, which we can think of as maps from an $n$-dimensional triangle into a space, and various morphisms $\delta_i: K_n \to K_{n-1}, \sigma_i: K_{n-1} \to K_n$ telling us how the triangles fit together.

Following \cite{Riehl_elementary},

\begin{itemize}

\subsection{November 29th: \(\infty\)-categories}

\subsection{December 6th: join and slices }  

\subsection{November 29th: Homotopical algebra and Model Categories}

\subsection{December 6th: The Quillen and Joyal model structures on Simplicial Sets}

\subsection{December 13th:}

\subsection{December 20th:}

\subsection{January 10th:}

\subsection{January 17th:}

\subsection{January 24th:}

\subsection{January 31th:}

\subsection{February 7th:}

\subsection{February 14th:}




\end{document}